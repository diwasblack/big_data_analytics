\documentclass[12pt]{article}
\usepackage[utf8]{inputenc}
\usepackage{setspace}
\usepackage{times}
\usepackage{cite}
\usepackage{graphicx}
\usepackage{listings}
\usepackage{hyperref}

\title{Monkey Species Identifier\\ \normalsize Project 2 \\ \small Big Data Analytics (CS 696-16)}
\author{Diwas Sharma \\ A25264728}
\date{October 22, 2018}

\singlespace

\begin{document}

\maketitle
\newpage

\tableofcontents
\newpage

\section{Introduction}
Constructing a classifier for automated categorization of images is not an easy task. It is further complicated by
the fact that each point in the image has some information about the color along with its spatial information. The aim of this
project is to construct a image classifier for identifying the species of a monkey.

\subsection{Dataset}
The dataset that was used in this project was obtained from kaggle \footnote{\url{https://www.kaggle.com/slothkong/10-monkey-species}}.
The dataset consists of colored images of 10 monkey species which are distributed as shown in table \ref{tbl:dataset}. 

\begin{table}[ht]
\centering
\begin{tabular}{ l c c c r }
\hline
Class                        & Training samples & Test samples \\
\hline
mantled\_howler              & 131              & 26 \\
patas\_monkey                & 139              & 28 \\
bald\_uakari                 & 137              & 27 \\
japanese\_macaque            & 152              & 30 \\
pygmy\_marmoset              & 131              & 26 \\
white\_headed\_capuchin      & 141              & 28 \\
silvery\_marmoset            & 132              & 26 \\
common\_squirrel\_monkey     & 142              & 28 \\
black\_headed\_night\_monkey & 133              & 27 \\
nilgiri\_langur              & 132              & 26 \\
\hline
& 1098 & 272 \\
\end{tabular}
\caption{Data distribution}
\label{tbl:dataset}
\end{table}

\subsection{Data Augmentation}
Since most of the classifier models generalize better when trained on larger dataset,
it is better to gather more data when possible. The original images on the dataset can be used
to generate new images as follows,

\begin{itemize}
    \item \textbf{Horizontal Flip:} Since flipping a image horizontally should not change the class assignment, it can be safely used to augment the dataset.
    \item \textbf{Rotation:} Another way to augment the dataset would be randomly rotating the image by a small degree $[-5, 5]$.
\end{itemize}

By randomly using the above techniques, the original dataset was used to create a total of 3294 image for training and 
816 image for testing.

\subsection{Data preprocessing}
Since the original images had varying image size which could be even as large as 1900 by 2000 pixels, the images were resized to a fixed image size of
200 by 200 pixels. The 8 bit unsigned integer representation of the images were then converted to the 32 bit floating point representation that could be used with
classifiers.

\section{Methods}

\subsection{Feature Extraction}
The Xception\cite{chollet2017xception} network initialized with weights trained on ImageNet dataset was used to extract meaningful features from an image.
The network would take the $200 * 200 * 3$ image as input and generate a $7 * 7 * 2048$ tensor which could be used as features for classification algorithms.

\subsection{Classification Model}
A three layer neural network was used as the classifier model for identifying the monkey species. The architecture of the neural network model is listed below.

\begin{itemize}
    \item Hidden Layer
    \begin{itemize}
        \item Units: 512
        \item Activation: ReLu
        \item Regularization: Dropout (rate=0.5) 
    \end{itemize}

    \item Output Layer
    \begin{itemize}
        \item Units: 10
        \item Activation: Softmax
    \end{itemize}
\end{itemize}

\subsection{Training and Testing}
The augmented training dataset is used to train the model after extracting the features using the Xception network and
the augemented test dataset is used to evaluate the performance of the classification model on the training dataset.

\section{Results}
The performance of the classifier on the test set is shown in table \ref{tbl:test_performance}.

\begin{table}[ht]
\centering
\begin{tabular}{ l c c c r }
\hline
Class & Precision & Recall & Fscore & Support \\
\hline
mantled\_howler & 0.98717949 & 0.98717949 & 0.98717949 & 78 \\
patas\_monkey & 0.97530864 & 0.94047619 & 0.95757576 & 84 \\
bald\_uakari & 1.0 & 0.43209877 & 0.60344828 & 81 \\
japanese\_macaque & 0.97333333 & 0.81111111 & 0.88484848 & 90 \\
pygmy\_marmoset & 0.9625 & 0.98717949 & 0.97468354 & 78 \\
white\_headed\_capuchin & 1.0 & 0.9047619 & 0.95 & 84 \\
silvery\_marmoset & 0.47852761 & 1.0 & 0.6473029 & 78 \\
common\_squirrel\_monkey & 0.70338983 & 0.98809524 & 0.82178218 & 84 \\
black\_headed\_night\_monkey & 1.0 & 0.43209877 & 0.60344828 & 81 \\
nilgiri\_langur & 1.0 & 0.96153846 & 0.98039216 & 78 \\
\hline
\end{tabular}
\caption{Test performance of the model}
\label{tbl:test_performance}
\end{table}

\section{Conclusion}
The classifier performs resonably well for most classes except for the "bald uakari", "silvery marmoset", and "black headed night monkey" classes.
The Fscore for those classes are quite low(around 0.6) compared to Fscore of other classes which are around 0.9.

\newpage
\bibliography{report}
\bibliographystyle{ieeetr}
\newpage

\section*{Appendix}
% \addtocounter{section}{1}

\subsection{Source Code}
\lstinputlisting[language=python]{"monkey_classifier.py"}

\subsection{Libraries Used}

\begin{itemize}
    \item numpy \url{http://www.numpy.org/}
    \item keras \url{https://keras.io/}
    \item scikit-learn \url{http://scikit-learn.org/stable/}
\end{itemize}

\end{document}

\end{document}
