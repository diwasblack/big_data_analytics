\documentclass[12pt]{article}
\usepackage[utf8]{inputenc}
\usepackage{setspace}
\usepackage{times}
\usepackage{cite}
\usepackage{graphicx}
\usepackage{listings}
\usepackage{hyperref}

\title{Human Activity Recognition Using Smartphones \\ \normalsize Project 1 \\ \small Big Data Analytics (CS 696-16)}
\author{Diwas Sharma \\ A25264728}
\date{September 17, 2018}

\singlespace

\begin{document}

\maketitle
\newpage

\tableofcontents
\newpage

\section{Introduction}
Smartphone nowadays contains a number of sensor such as accelerometer, gyroscope, light sensor, proximity sensor, fingerprint sensor, magnetometer sensor, etc. The goal of the project
is to predict the activity of a person from the sensor readings. The human activity recognition dataset\cite{anguita2013public} is a labelled dataset that has the readings from accelerometer
and gyroscope.

The dataset contains 10299 instances with 561 feature each which are labelled as one of the activity among "WALKING, "WALKING\_UPSTAIRS", "WALKING\_DOWNSTAIRS", "SITTING", "STANDING" and "LAYING".

\section{Methods}

\subsection{Data Acquisition}
The human activity recognition dataset can be obtained from either UCI machine learning repository \footnote{https://archive.ics.uci.edu/ml/datasets/human+activity+recognition+using+smartphones}
or kaggle \footnote{https://www.kaggle.com/uciml/human-activity-recognition-with-smartphones}.

\subsection{Models}
The scikit-learn\footnote{http://scikit-learn.org/stable/index.html} python library was used to implement the classifier models. The models implemented are as follows,

\begin{itemize}
    \item Regularized Logistic Regression \cite{hosmer2013applied}
    \item Support Vector Machine \cite{cortes1995support}
    \item Random Forest \cite{liaw2002classification}
\end{itemize}

\subsection{Training and Testing}
At the first, the dataset is divided into training and test set using a 2/3 : 1/3 split. Then, 5 fold cross validation technique is used to select the best hyperparameters for the
models. The models are then trained on the whole training set using the best hyperparameters obtained from the cross validation step.

The models are then evaluated on the test set to contrast their performance.

\section{Results}
\subsection{Data Visualization}
It is better to first visualize the data in order to understand it. The t-SNE\cite{maaten2008visualizing} algorithm can be used to reduce a high dimensional data to either 2 or 3 dimensions so that it can visulized
using plotting library such as matplotlib\cite{hunter2007matplotlib}.

\begin{figure}[ht]
  \centering
  \includegraphics[width=\linewidth]{dataset.png}
  \caption{Visualization of dataset using t-SNE algorithm}
  \label{fig:dataset}
\end{figure}

\subsection{Performance}
The test accuracies of the models for the dataset is shown in Table \ref{tbl:test_accuracies}.

\begin{table}[ht]
\centering
\begin{tabular}{ l c r }
\hline
Model & Test Accuracy \\
\hline
Regularized Logistic Regression & 97.48\\
Support Vector Machine & 98.40 \\
Random Forest & 95.88 \\
\hline
\end{tabular}
\caption{Test accuracies of models}
\label{tbl:test_accuracies}
\end{table}

\section{Conclusion}
Thus it shows that for the given dataset it is possible to predict the human activity with high accuracy based on the readings from accelerometer and gyro sensor embedded
within a smartphone.

\newpage
\bibliography{report}
\bibliographystyle{ieeetr}
\newpage

\section*{Appendix}
% \addtocounter{section}{1}

\subsection{Source Code}
\lstinputlisting[language=python]{"human_activity.py"}

\subsection{Libraries Used}

\begin{itemize}
    \item numpy \url{http://www.numpy.org/}
    \item pandas \url{https://pandas.pydata.org/}
    \item matplotlib \url{https://matplotlib.org}
    \item scikit-learn \url{http://scikit-learn.org/stable/}
\end{itemize}

\end{document}
